\documentclass{article}
\usepackage{graphicx} % Required for inserting images
\usepackage{amsmath}
\usepackage[margin=2cm]{geometry}
\usepackage{gensymb}



\title{Multi-Variable Control Systems Project Update 1}
\author{Alek Krupka}
\date{February 2026}

\begin{document}
    \maketitle

    \section{Introduction}\label{sec:introduction}

    This project update will mainly be focused on discussing the project
    description along with a modification that will allow the project to be used
    for a SISO controller design.
    Additionally, a SISO controller is proposed to control this modified system.

    \section{Project Motivation}\label{sec:motivation}

    The goal for the semester-long project is to develop a control system
    for a 2-propeller drone.
    This is a simplification made to the standard four propeller drone which
    reduces the degrees of freedom experienced by the input.
    While this simplication confines the drone to a single plane (the plane formed
    by the z-axis and the axis along which the drone arms form), it still allows
    for the design of a multi-variable control system.
    This project was selected to gain experience with controlling a slow-reacting
    and non-linear system.
    The details of the system's non-linearity will be discussed later in \ref{subsec:system-description}.
    Even with only having two propellers, the system still maintains these properties, meaning
    the simplification to a two-propeller drone does not negate this motivation.

    \section{Project Definition}\label{sec:description}

    \subsection{System Description}\label{subsec:system-description}

    To begin, we define a drone with just two propellers.
    This allows us to ignore pitch (rotation around the x-axis) and yaw (rotation around the z-axis).
    However, this restricts the drone to just rotating around its y-axis and moving along the x and z-axis.
    For this project, that will be sufficient though.
    Below, we derive the equations for the output based on the input (speed of propeller for each drone).

    We treat this drone as a rigid body with two motors along the x-axis with distance $r$ away from the center of mass.
    Each motor $i$ produces a linear acceleration in the $z$ direction from the drone's perspective.
    This thrust force is equivalent to the following.

    \begin{align}
        F_{thrust} = k_f\omega^2
    \end{align}

    where $k_f$ is the thrust coefficient for all propellers.\\

    Analyzing the impact on the body frame, we see that the acceleration from the body
    frame is the following.

    \begin{align}
        a_{body} = \begin{bmatrix} 0 \\ 0 \\ \frac{\sum_{k=0}^1(k_f\omega_i^2)}{m} \end{bmatrix}
    \end{align}

    Now we must calculate the torque applied to the drone.
    To calculate the angular acceleration due to the thrusts from each propeller,
    we need to calculate the moment of inertia for the entire drone.
    For this section, we will solve this symbolically and will list its assumed value in section~\ref{subsec:model-uncertainty}.
    For roll, we say that the acceleration around the y-axis is the following.

    \begin{align}
        \alpha_y = \frac{l*k_f(\omega_1^2 - \omega_2^2)}{I}
    \end{align}

    To get the real-world directional accelerations though, we need to translate
    the body-frame accelerations into the real-world coordinate system.
    This is done using the rotation matrix listed below,

    \begin{align}
        R_y(\theta) = \begin{bmatrix} \cos(\theta) & 0 & \sin(\theta) \\ 0 & 1 & 0 \\
            -sin(\theta) & 0 & \cos(\theta) \end{bmatrix}
    \end{align}

    Multiplying this rotation matrix by the directional accelerations produces the real-world
    acceleration value.

    This gives us the acceleration space definition shown below.

    \begin{align}
        \begin{bmatrix} a_x \\ a_y \\ a_z \end{bmatrix} &= R_y(\theta) a_{z, body}\\
        &= \begin{bmatrix} \sin(\theta) \\ 0 \\ \cos(\theta) \end{bmatrix} \frac{\sum_{k=0}^1(k_f\omega_i^2)}{m}
    \end{align}

    Afterward, gravity accounts for a downward acceleration of $-9.81 m/s^2$ giving us the final real-world
    frame definition of the accelerations.

    \begin{align}
        a_x = \sin(\theta) \frac{\sum_{k=0}^1(k_f\omega_i^2)}{m}\\
        a_z = \cos(\theta) \frac{\sum_{k=0}^1(k_f\omega_i^2)}{m} - 9.81
    \end{align}

    Using these equations, we are able to generate a space with six states.
    These states are the directional positions along with rotation and each
    of their first derivatives.
    The equations for each state are below.

    \begin{align}
        \dot{x} = v_x\\
        \dot{v_x} = a_x = \sin(\theta) \frac{\sum_{k=0}^1(k_f\omega_i^2)}{m}\\
        \dot{z} = v_z\\
        \dot{v_z} = a_z = \cos(\theta) \frac{\sum_{k=0}^1(k_f\omega_i^2)}{m} - 9.81\\
        \dot{\theta} = \omega_y\\
        \dot{\omega_y} = \alpha_y = \frac{l*k_f(\omega_1^2 - \omega_2^2)}{I}
    \end{align}

    However, this system is nonlinear due to the acceleration being proportional to the squared motor speed.
    This linearization will be covered later due to being able to linearize differently depending on the operating
    point selected at the time.
    The final state space will look something like this though.

    Start by defining the state space $x$ as follows,

    \begin{align}
        x = \begin{bmatrix} x, v_x, z, v_z, \theta, \omega_y \end{bmatrix}^T\\
    \end{align}

    Thus, our state space will look something like the following.

    \begin{align}
        \dot{x} = \begin{bmatrix} 0 & 1 & 0 & 0 & 0 \\ 0 & 0 & 0 & 0 & \Phi_1 & 0 \\
        0 & 0 & 0 & 1 & 0 & 0 \\ 0 & 0 & 0 & 0 & \Phi_2 & 0 \\
        0 & 0 & 0 & 0 & 0 & 1 \\ 0 & 0 & 0 & 0 & 0 & 0 \end{bmatrix} x +
        \begin{bmatrix} 0 & 0 \\ \rho_1 & \rho_2 \\ 0 & 0 \\ \rho_3 & \rho_4 \\
        0 & 0 \\ \rho_5 & \rho_6 \end{bmatrix} \begin{bmatrix} \omega_1 & \omega_2 \end{bmatrix}
    \end{align}

    where $\Rho_k$ and $\phi_j$ are linearization constants.

    These variables are filled in once an operating point has been determined.

    \subsection{Model Uncertainty}\label{subsec:model-uncertainty}

    The main uncertainty in the model comes mainly from the moment of inertia
    along with the value of the thurst coefficient $k_f$.
    There is additional uncertainty with the mass of the drone.
    While the mass can be measured, there will be some error and the mass will not always be the same

    For the thrust coefficient a reasonable value we will be assuming is

    \begin{align}
        k_f = 10^{-5} \pm 5%
    \end{align}

    For this project, we will be assuming a mass of

    \begin{align}
        0.5 \pm 0.1 kg
    \end{align}

    We will assume the moment of inertia for this two-propeller drone is a rod.
    This is because the two-propeller drone is just a rod with motors on the end
    While there is slight weighting issues we will account for this in the model uncertainty.
    The moment of intertia for the rod is

    \begin{align}
        I_{rod} = \frac{4}{3}ml^2
    \end{align}

    For this project, assume each motor is roughly $10^{-2}m$ away from the center.
    Using these measurements, we get a moment of

    \begin{align}
        I_{rod} \approx \frac{2}{3} \times 10^{-4}
    \end{align}

    This moment is assumed to have an error of around $10^{-4}$ when moment error and
    previously defined mass errors are taken into consideration.

    \subsection{Exogenous Input}\label{subsec:exogenous-input}

    Characteristics of exogenous input to the system include wind and measurement noise from the GPS.\\
    \\
    For the GPS, real-world sensors typically do not measure to more than 1.5m accuracy.
    In this project, we will assume that the noise coming from every measurement is uniformly distributed
    and that all RVs are i.i.d.\\
    \\
    Wind noise in the system can be simplified to cause acceleration of the drone in a random direction.
    While in the real-world wind patterns can be predicted and are not random, for this project we will assume that wind
    pushes the drone in a random direction every sample.
    The acceleration in each direction is normally and identically distributed with a variance of $1m/s^2$.
    \\
    For the final SISO system, the voltage to motor speed will not be known and a separate control loop will need to be used
    for keeping the motor at the correct speed.

    \subsection{Control Objective}\label{subsec:control-objective}
`   \\
    The objective for developed controller will be to have the drone go from an initial position $p_i$ to
    a final position $p_f$ while keeping the following constraints in mind.\\
    \begin{itemize}
        \item Overshoot: The controllers should not overshoot the magnitude of the final position by
            greater than $2\%$ of the destination.
            For example, if going to a final position of $(x = 1000, z = 1000)$
            going from point $(x=0, z=0)$, let $X_t$ be the amount it goes past 1000 on the $x-axis$ and
            $Z_t$ be the amount it goes past $1000$ on the z-axis at time $t$.
            In this case, at no time $t$ should $\sqrt{X_t^2 + Z_t^2}$ be greater than $\sqrt{1000^2}$.
        \item Steady state error: The steady state error for the $z$ position should be no greater than $0.1\%$ of the
            z-setpoint.
            Meanwhile, the steady state error for the $x$ position should be at most $2m$ from the target.
        \item Steady State Noise: The noise should be at most $0.25\%$ on the z-axis and at most $4m$ from the target
            on the x-axis.
        \item Phase Margin: Any controller for this system should have at least $45 \degree$ of phase margin.
        \item Max rotation: The drone should not rotate by more than $30 \degree$ in either direction at any point in it's flight.
        \item Max velocity: $||v_z + v_x||$ will be limited to $9m/s \approx 20 mph$ in the final design.
    \end{itemize}

    These are all control objectives for the final design of the final controller.

    \section{SISO Subsystem Description for Project}\label{sec:siso-description}

    The Single-Input-Single-Output modification can be made to the drone.
    Assume that both propellers spin at exactly the same speed.
    Now, the only output is the z-position measurement and the only input is the desired propeller speed.
    With these restrictions, the drone can be treated as a SISO control system which can be modeled.
    In this report, a SISO controller will be designed for this subsystem.

    \section{Proposed Controller Designs for SISO Subsystem}\label{sec:proposed-siso-controllers}

    \section{Results}\label{sec:siso_results}






\end{document}
